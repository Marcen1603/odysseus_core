\documentclass{article}

\usepackage{ngerman}
\usepackage[latin1]{inputenc}

\title{Erste Anforderungen bzgl. PQL/Grammatik}
\author{Thomas Vogelgesang, Timo Michelsen}
\date{\today}

\begin{document}

	\maketitle

	\begin{itemize}
		\item PQL-Grammatik anpassen
		\begin{itemize}
			\item Broker-Operator muss als Quelle angesprochen werden k�nnen (f�r weitere Anfragen�)
			\item Broker-Operator muss in PQL als Operator zur Verf�gung stehen
			\item Prediction-Operator muss in PQL als Operator zur Verf�gung stehen
			\item Assoziation-Operator muss in PQL als Operator zur Verf�gung stehen
			\item Filterung-Operator muss in PQL als Operator zur Verf�gung stehen
			\item Quellen, die die Sensorendaten liefern, m�ssen in PQL ansprechbar sein
			\item Senken m�ssen in PQL verwendet werden k�nnen (insbesondere die UDP-Senke)
			\item Dem Broker-Operator m�ssen in PQL beliebig viele Eing�nge �bergeben werden k�nnen
		\end{itemize}		
	\end{itemize}

\end{document}