%%Hier wird das Dokument definiert

\documentclass{scrartcl}  %Format und Stil

\usepackage{../style/wrstyle}
	
% Einstellungen f\"{u}r die "offizielle" Titelseite
\title{Projektgruppe StreamCars} % Titel der Dissertation
\subtitle{w�chentliche Arbeitsberichte}
\author{Timo Michelsen} % Name des Autors
\date{\today}
\surveyor{Andr� Bolles}
\term{Sommersemester 2010 - Wintersemester 2010/11}

%%Hier wird das Dokument definiert
\begin{document}

	\maketitle
	
	\newpage
	
	\tableofcontents
	
% Wochenbericht KW 20
\week

\begin{workDone}
	\begin{itemize}
		\item Projektmanagement
			\begin{itemize}
				\item Mittwochs mit Hauke getroffen
				\item Neueste Fortschritte in MSProject eingetragen (Gantt-Diagramm usw.)
				\item Dabei erste Aufgaben auf die Teammitglieder verteilt
				\item Kleine TOP-Liste angefertigt, was unbedingt am Donnerstag und Freitag besprochen werden muss
			\end{itemize}
		\item Projektgruppe
			\begin{itemize}
				\item Aufgaben verteilt
				\item Erste Implementierungsversuche im Zusammenhang mit Transformationsregeln
					\begin{itemize}
						\item Verst�ndnis f�r die Umwandlungsprozedur von logischen zu physischen Operatoren entwickeln
						\item Frage beantworten, was alles daf�r in Odysseus definiert/implementiert werden muss
						\item Dazu einen Testoperator geschrieben, welcher die eintreffenden Datenelemente in der Konsole ausgibt und unver�ndert weiterleitet
						\item PQL-Grammatik erweitert
						\item Erste Schritte mit der Regel-Sprache von Drools
					\end{itemize}
			\end{itemize}
		\item Sonstiges
			\begin{itemize}
				\item Hauke und Sven mit eigenen C++-Kenntnissen unterst�tzt
			\end{itemize}
	\end{itemize}
\end{workDone}

\begin{workProblems}
	\begin{itemize}
		\item Projektmanagement
			\begin{itemize}
				\item Grunds�tzlich schwierig, Umfang der Aufgaben abzusch�tzen
				\item 
			\end{itemize}
		\item Schwierig, sich in die Tranformationsregeln einzuarbeiten
		\item Regel-Sprache ist sehr m�chtig und ausdrucksstark
	\end{itemize}
\end{workProblems}

\begin{workToDo}
	\begin{itemize}
		\item Weiter das Thema Transformation untersuchen und verstehen
		\item �ber PQL den logischen Testoperator in einen physischen umwandeln lassen (programmatisch)
		\item Da es zu umfangreich ist, werde ich Thomas dazu holen
	\end{itemize}
\end{workToDo}

% Wochenbericht KW 21
\week
\begin{workDone}
	\begin{itemize}
		\item Einen Testoperator vollst�ndig implementiert und getestet
		\item Vollst�ndigen Prozess zum Hinzuf�gen eines neuen Operators in Odysseus verstanden. Das Wissen kann dann sp�ter auf die PAF-Operatoren �bertragen werden.
		\item PQLHack entsprechend erweitert, sodass der TESTOP in Queries eingesetzt werden kann.
	\end{itemize}
\end{workDone}

\begin{workProblems}
	\begin{itemize}
		\item Keine
	\end{itemize}
\end{workProblems}

\begin{workToDo}
	\begin{itemize}
		\item Sich als Projektleiter wieder auf den neuesten Stand der einzelnen Aufgaben bringen lassen.
		\item Neue Aufgaben suchen und evtl. verteilen.
	\end{itemize}
\end{workToDo}

% Wochenbericht KW 22
\week
\begin{workDone}
	\begin{itemize}
		\item Braunschweig: Seminarvortrag abgeschlossen
		\item Grammatik f�r \texttt{CREATE SENSOR(...)} in CQL eingebaut, aber noch keine Java-Logik vorhanden
		\item Dabei sind wir mittels Attribute, Records und Lists in der Lage, objektrelationale Schemata darzustellen.
	\end{itemize}
\end{workDone}

\begin{workProblems}
	\begin{itemize}
		\item Der Versuch, \texttt{CREATE SENSOR(...)} �ber PQLHack zu realisieren, scheiterte an Komplexit�t und Umfang (viel Zeit verschwendet).
		\item Dokumentation von CQL mangelhaft bzw. nicht vorhanden. Es musste viel Code gelesen und verstanden werden.
	\end{itemize}
\end{workProblems}

\begin{workToDo}
	\begin{itemize}
		\item \texttt{CREATE SENSOR(...)} eine Java-Logik verpassen, sodass wir in der Lage sind, aus der Grammatik Parameter, Attributschemata usw. auslesen zu k�nnen.
		\item Dabei ist es wichtig, viel zu Testen.
		\item Mit Andr� abstimmen, ob die \texttt{CREATE SENSOR(...)}-Syntax alles ben�tigte abdeckt.
		\item Sind die Tests viel versprechend und Andr� zufrieden, dann die neue Quelle in Odysseus registrieren (evtl. �ber \texttt{DataDictonary})
	\end{itemize}
\end{workToDo}

% Wochenbericht KW 23
\week
\begin{workDone}
	\begin{itemize}
		\item Erste Java-Implementierung der CQL-Grammatik eingebaut
		\item K�nnen die objektrelationale Schemadefinition in \texttt{SDFAttributes} umwandeln
		\item CQL-Grammatik erstellt.
		\item Wolf auf den neuesten Stand gebracht.
		\item Tests f�r Visitor geschrieben und ausgef�hrt $\rightarrow$ erfolgreich!
		\item Versuch, Transformationsregeln f�r den Quellenoperator zu schreiben $\rightarrow$  fehlschlag!
	\end{itemize}
\end{workDone}

\begin{workProblems}
	\begin{itemize}
		\item \texttt{SDFAttribute} u. �. besitzen keine Dokumentation. Viel Code lesen und verstehen.
		\item Relikte alter Implementierungen gefunden, die unn�tz sind, aber den Leser durcheinander gebracht hatten (mit Jonas gel�st)
		\item Drools hilft �berhaupt nicht bei Problemen! Erhalte immer komische Fehlermeldungen beim Compilieren von drl-Dateien.
	\end{itemize}
\end{workProblems}

\begin{workToDo}
	\begin{itemize}
		\item Transformationsregeln f�r den Quellenoperator schreiben.
		\item Tests
	\end{itemize}
\end{workToDo}

% Wochenbericht KW 24
\week
\begin{workDone}
	\begin{itemize}
		\item Transformationsregel f�r den Quellenoperator geschrieben.
		\item Konnten \texttt{CREATE STREAM(...)} erfolgreich ausf�hren.
		\item Quellenzugriff mit \texttt{SELECT * FROM superSensor} erfolgreich getestet.
		\item Erste Version des Productbacklog angefertigt. Anderen Bescheid gegeben, sich damit zu befassen und Anforderungen einzutragen.
		\item Die Quelle muss den Bytestrom je nach angegebenen Schema (aus PQL) auslesen. Dazu haben ich, Wolf und Thomas die JDVEAccessMVPO entsprechend angepasst. Die Funktionsweise ist noch nicht getestet.
	\end{itemize}
\end{workDone}

\begin{workProblems}
	\begin{itemize}
		\item Transformationsregeln k�nnen nur auf OSGi-Plugins zugreifen. Nicht auf Eclipse-Plugins. Hauke und Sven haben f�r den physichen Quellenoperator ein Eclipse-Plugin erstellt, sodass die Transformationsregel merkw�rdige Fehlermeldungen ausgab. Es dauerte lange, den Fehler zu finden und zu beheben.
		\item Wir haben nur einen kurzen Zeitraum f�r das Productbacklog.
	\end{itemize}
\end{workProblems}

\begin{workToDo}
	\begin{itemize}
		\item Neue JDVEAccessMVPO testen. Ggfs. muss die Sender-Seite (JDVE) noch angepasst werden. Dabei wird dann Hauke helfen. 
		\item Projektmanagement: Sich auf den neuesten Stand bringen lassen. Danach das Gantt-Diagramm anpassen (mit Hauke).
		\item Projektmanagement: Eventuell muss gekl�rt werden, wann wir freie Wochen einbauen. Klausuren stehen an.
	\end{itemize}
\end{workToDo}


\end{document}
