%%Hier wird das Dokument definiert

\documentclass{scrartcl}  %Format und Stil

\usepackage{../style/wrstyle}
	
% Einstellungen f\"{u}r die "offizielle" Titelseite
\title{Projektgruppe StreamCars} % Titel der Dissertation
\subtitle{w�chentliche Arbeitsberichte}
\author{Nico Klein} % Name des Autors
\date{\today}
\surveyor{Andr� Bolles}
\term{Sommersemester 2010 - Wintersemester 2010/11}

%%Hier wird das Dokument definiert
\begin{document}

	\maketitle
	
	\newpage
	
	\tableofcontents
	
% Wochenbericht KW 20
\week

\begin{workDone}
	\begin{itemize}
		\item Prediction-Algorithmus
		\begin{compactitem}
			\item Der Algorithmus wurde ausgew�hlt.
			\item Der Algorithmus wurde implementiert.
			\item Es wurde ein Test implementiert der jedoch noch ausgiebig durchgef�hrt werden muss.
		\end{compactitem}
		\item Erstellung der Pr�sentationsfolien.
		\item Erkl�rung der Funktionalit�t von logischen und physischen Operatoren.
	\end{itemize}
\end{workDone}

\begin{workProblems}
	\begin{itemize}
		\item JUnit funktioniert noch nicht.
	\end{itemize}
\end{workProblems}

\begin{workToDo}
	\begin{itemize}
		\item Prediction-Algorithmus
		\begin{compactitem}
			\item JUnit muss funktionsf�hig gemacht werden.
			\item Der Algorithmus muss ausgiebig getestet werden.
		\end{compactitem}
		\item Assoziation-Algorithmus
		\begin{compactitem}
			\item Der Algorithmus muss ausgew�hlt werden.
			\item Der Algorithmus muss implementiert werden.
			\item Der Algorithmus muss getestet werden.
		\end{compactitem}
	\end{itemize}
\end{workToDo}

% Wochenbericht KW 21
\week
\begin{workDone}
	\begin{itemize}
		\item JUnit ist funktionsf�hig
		\item Prediction-Algorithmus wurde getestet und ist funktionsf�hig
		\item Als Assoziationsalgorithmus wurde Nearest-Neighbor ausgew�hlt
	\end{itemize}
\end{workDone}

\begin{workProblems}
	\begin{itemize}
		\item Es gab Probleme beim einbinden einer mathematischen Library
		\item Dadurch konnte der Assoziations-Algorithmus nicht fertiggestellt werden
	\end{itemize}
\end{workProblems}

\begin{workToDo}
	\begin{itemize}
		\item Assozitations-Algorithmus
		\begin{itemize}
			\item Der Algorithmus muss fertiggestellt werden
			\item Der Algorithmus muss getestet werden
		\end{itemize}
		\item Filterungs-Algorithmus
		\begin{itemize}
			\item Der Algorithmus muss ausgew�hlt werden
			\item Der Algorithmus muss implementiert werden
			\item Der Algorithmus muss getestet werden
		\end{itemize}
	\end{itemize}
\end{workToDo}

\week
\begin{workDone}
	\begin{itemize}
		\item Der Assoziationsalgorithmus Nearest Neighbor wurde implementiert
		\item Der Assoziationsalgorithmus Nearest Neighbor wurde getestet
		\item Fahrt nach Braunschweig zum DLR
	\end{itemize}
\end{workDone}

\begin{workProblems}
	\begin{itemize}
		\item Es gab keine Probleme
	\end{itemize}
\end{workProblems}

\begin{workToDo}
	\begin{itemize}
		\item Algorithmen allgemein
		\begin{itemize}
			\item Die Algorithmen m�ssen parametrisierbar gemacht werden
		\end{itemize}
		\item Filterungs-Algorithmus
		\begin{itemize}
			\item Der Algorithmus muss ausgew�hlt werden
			\item Der Algorithmus muss implementiert werden
			\item Der Algorithmus muss getestet werden
		\end{itemize}
	\end{itemize}
\end{workToDo}

\week
\begin{workDone}
	\begin{itemize}
		\item �berlegungen zur Parametrisierung der Operatoren.
		\item �berlegungen zur Generik der Operatoren durchgef�hrt.
		\item Zusammenhang zwischen AOs und POs und den Transformatioinsregeln verstanden.
		\item Initialisierung des Kontextmodells �berarbeitet (Vorversion von Benny)
	\end{itemize}
\end{workDone}

\begin{workProblems}
	\begin{itemize}
		\item Es gab keine Probleme
	\end{itemize}
\end{workProblems}

\begin{workToDo}
	\begin{itemize}
		\item Die Operatoren zu den PAF-Algorithmen m�ssen implementiert werden.
		\item Auf Basis des Verst�ndnisses zu AOs, POs und den Transformationsregeln k�nnen sowohl die AOs, als auch die POs implementiert werden.
		\item Testen der Operatoren (In Absprache mit den Gruppenmitgliedern, welche die Transformationsregeln bearbeiten)
	\end{itemize}
\end{workToDo}

\week
\begin{workDone}
	\begin{itemize}
		\item Weitere Diskussion zur Parametrisierung der Operatoren.
		\item Aufteilung der Assoziation in einzelne Schritte.
		\begin{itemize}
			\item 1. Schritt: Zun�chst wird eine Matrix der neu angekommenen und schon existierenden (zu vergleichenden) Objekte erstellt.
			\item 2. Schritt: Hier k�nnen beliebig viele Bewertungsfunktionen, wie z.B. Mahalanobis oder Nearest Neighbor, durchgef�hrt werden. Dabei wird die zuvor erstellte Matrix durch die Bewertungsfunktionen ver�ndert. Die Matrix enth�lt dann die Wahrscheinlichkeiten, dass ein neu angekommenes Objekt zu den existierenden passen kann.
			\item 3. Schritt: Endg�ltige Zuordnung der neuen zu den existierenden Objekten anhand der Wahrscheinlichkeiten der Matrix.
		\end{itemize}
		\item Der Prediction-Operator bekommt durch einen PredictionAssign-Operator die jeweilig ausgew�hlte Prediction-Function �bergeben.
		\item Die Filterung ben�tigt die Auswahl der Prediction-Function (bspw. aufgrund einer bestimmten Wetterbedingung), um die passende Filterungsfunktion ausw�hlen zu k�nnen.
		\item An Product Backlog mitgearbeitet.
	\end{itemize}
\end{workDone}

\begin{workProblems}
	\begin{itemize}
		\item Die Operatoren konnten nicht wie zuerst gedacht implementiert werden.
		\item Die Parametrisierung der Operatoren war nicht ausreichend.
	\end{itemize}
\end{workProblems}

\begin{workToDo}
	\begin{itemize}
		\item Implementierung der einzelnen Funktionalit�ten des Assoziationsalgoritmus:
		\begin{itemize}
			\item Meine Aufgabe ist es zun�chst die Funktion zu erstellen, welche die Matrix erstellt, welche f�r die weiteren Schritte der Assoziation ben�tigt wird.
		\end{itemize}
	\end{itemize}
\end{workToDo}

\end{document}
