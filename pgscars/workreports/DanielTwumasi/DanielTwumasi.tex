%%Hier wird das Dokument definiert

\documentclass{scrartcl}  %Format und Stil

\usepackage{../style/wrstyle}
	
% Einstellungen f\"{u}r die "offizielle" Titelseite
\title{Projektgruppe StreamCars} % Titel der Dissertation
\subtitle{w\"ochentliche Arbeitsberichte}
\author{Daniel Twumasi} % Name des Autors
\date{\today}
\surveyor{Andr\'e Bolles}
\term{Sommersemester 2010 - Wintersemester 2010/11}

%%Hier wird das Dokument definiert
\begin{document}

	\maketitle
	
	\newpage
	
	\tableofcontents
	
% Wochenbericht KW 20
\week

\begin{workDone}
	\begin{itemize}
		\item Pras\"entation und Ausarbeitung ausarbeiten.
		\item In Odysseus einarbeiten.
	\end{itemize}
\end{workDone}

\begin{workProblems}
	\begin{itemize}
		\item Krankgeschrieben von Donnerstag bis Samstag.
		\item Deshalb keine Einarbeitung.
	\end{itemize}
\end{workProblems}

\begin{workToDo}
	\begin{itemize}
		\item Einarbeitung in Odysseus.
		\item Implementierung des Kalmanfilers.
	\end{itemize}
\end{workToDo}

% Wochenbericht KW 21
\week
\begin{workDone}
	\begin{itemize}
		\item Zusammenarbeit bei Implementierung des Gating und Assoziationsoperators.
		
\end{itemize}
\end{workDone}

\begin{workProblems}
	\begin{itemize}
		\item Kalmanfiler aus Zeitgr\"unden noch nicht implementiert.
\end{itemize}
\end{workProblems}

\begin{workToDo}
	\begin{itemize}
		\item Einarbeitung in Odysseus.
		\item Kalmanfilter implementieren und testen
		\item Vorbereiten und halten der Pr\"asentation in Braunschweig.	
	\end{itemize}
\end{workToDo}

% Wochenbericht KW 22
\week
\begin{workDone}
	\begin{itemize}
		\item M\"ogliche Einteilung des Filterungsprozesses (Filter) in geeignete Teilschritte bzw. Operatoren identifiziert.
		\item Zusammenarbeit bei Implementierung des Gating und Assoziationsoperators.
\end{itemize}
\end{workDone}

\begin{workProblems}
	\begin{itemize}
		\item Andere Filter wie z.B. Partikelfilter m\"ussen ggf. in andere Teilschritte eingeteilt werden. Daher Einarbeitung in diese Techniken n\"otig.
		\item Zusammenh\"ange in Odysseus noch nicht durchschaut.
\end{itemize}
\end{workProblems}

\begin{workToDo}
	\begin{itemize}
		\item Kalmanfilter implementieren und testen
		\item Zusammenarbeit bei Implementierung des Gating und Assoziationsoperators.
	\end{itemize}
\end{workToDo}


% Wochenbericht KW 23
\week
\begin{workDone}
	\begin{itemize}
		\item Einteilung in Teilschritte auf Grundlage des Bayes-Frameworks vorgenommen. \\
		      Einteilung in Teilschritte Initialisierung und Korrektur
		\item Beim Gating Operator die Mahalanobis Distanz bzgl. der Kovarianzmatrix verstanden.
		
\end{itemize}
\end{workDone}

\begin{workProblems}
	\begin{itemize}
		\item Diese Woche nicht gen\"ugend Zeit f\"ur die Implementierung des Filters verwendet. Daher n\"achste Woche mehr Zeit einteilen.
		\item Zusammenh\"ange in Odysseus noch nicht durchschaut.
\end{itemize}
\end{workProblems}

\begin{workToDo}
	\begin{itemize}
		\item Kalmanfilter auf Grundlage des Bayes-Frameworks implementieren und testen
		\item Zusammenarbeit bei Implementierung des Gating und Assoziationsoperators.
		\item Product Backlog erweitern
		\item Dokument zur Initialisierung des Kontextmodells durcharbeiten.
	\end{itemize}
\end{workToDo}


% Wochenbericht KW 24
\week
\begin{workDone}
	\begin{itemize}
		\item Product Backlog erweitert
		\item Dokument zur Initialisierung des Kontextmodells durchgearbeit bzw. besprochen.
		\item Zusammenarbeit bei Implementierung des Gating und Assoziationsoperators.
		\item In das Thema Datenfusion eingelesen
		\item In das Thema Change Point Detection eingelesen
		
\end{itemize}
\end{workDone}

\begin{workProblems}
	\begin{itemize}
		\item Kalman-Filter noch nicht implementiert.
		\item Zusammenh\"ange in Odysseus noch nicht durchschaut.
\end{itemize}
\end{workProblems}

\begin{workToDo}
	\begin{itemize}
		\item Konzept f�r Initialisierung (Track Initialisierung) und Korrektur des Kalmanfilters �berlegen
		\item Zusammenarbeit bei Implementierung des Gating und Assoziationsoperators.

	\end{itemize}
\end{workToDo}


% Wochenbericht KW 25
\week
\begin{workDone}
	\begin{itemize}
	\item Konzept f�r Gating und Assoziationsoperator besprochen. \\
	      Ergebnis: Nicht beide Schritte in einen Operator.
	\item Kalman Filter Gleichungen bzw. Korrektur Schritt besprochen. \\
	Ergebnis: Prozessmodell f�r Korrektur Schritt nicht n�tig
	\item Konzept f�r Speicherung der Distanzen bzw. Assoziationen besprochen. \\
	Ergebnis: Speicherung der Assoziation als Attribut des jeweiligen Objektes statt als Matrix
	\item Initialisierung im Filter besprochen. \\
	Ergebnis: Initialwert evtl mit Messwert gleichsetzen.
	\item Paper zum Thema Initialisierung gefunden und gelesen.
		
\end{itemize}
\end{workDone}

\begin{workProblems}
	\begin{itemize}
	\item 
\end{itemize}
\end{workProblems}

\begin{workToDo}
	\begin{itemize}
	\item

	\end{itemize}
\end{workToDo}
\end{document}