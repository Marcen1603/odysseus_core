\week
\begin{workDone}
	\begin{itemize}
		\item Neues Konzept f�r Assoziation �berlegt:
		\begin{itemize}
			\item Aufteilung in drei Hauptteile: Verbindungserzeugung, Verbindungsbewertung, Verbindungsauswahl.
			\item Hierzu drei Hauptoperatoren, je einen f�r jeden Hauptteil.
			\item Aufbau einer Assoziationsarchitektur: 1 - 1 ... n - 1 (Erzeugungsoperator - Bewertungsoperatoren - Auswahloperator).
			\item Hierdurch ist man flexibler als nur Gating und Assoziation zu benutzen.
			\item Der Erzeugungsoperator erstellt eine NxM-Matrix aus int-Werten wobei N die Anzahl an neuerkannten Objekten und M die Anzahl an alten, pr�dizierten Objekten ist. Als Initialeintrag kommt �berall 0 in die Matrix.
			\item Die Bewertungsoperatoren k�nnen im Bewertungsteil die Werte in der Matrix (Verbindungswahrscheinlichkeit) erh�hen oder niedriger setzen.
			\item Die Bwertungsstruktur kann z. B. durch Selektionsopertoren erweitert werden.
			\item Der Auswahloperator w�hlt nun die wahrscheinlichsten Verbindungen und ist auch f�r die Weiterleitung neuerkannter und nicht zugewiesener Objekte in den tempor�ren Zyklus zust�ndig.
			\item Der minimale Aufbau einer solchen Assoziationsstruktur w�re das, was Andr� in seiner Dissertation beschrieben hat: Erzeugungsoperator - Bewertungsoperator (Gating) - Auswahloperator (Assoziation).
		\end{itemize}
		\item Mit Nico und Benjamin intensiver �ber die Datenstruktur diskutiert.
		\item Interface f�r Bewertungsfunktionen erstellt.
	\end{itemize}
\end{workDone}

\begin{workProblems}
	\begin{itemize}
		\item Ideale Datenstruktur noch nicht klar.
		\item Probleme bei der Filterung: Nach Diskussion mit Daniel waren wir unsicher ob wir in der Filterung auch das Prozessmodell kennen m�ssen.
		\item Woher wissen wir, an welcher Stelle im relationalen Tupel etwas bestimmtes steht (welchen Index hat z. B. die Liste mit neuen Objekten?)
	\end{itemize}
\end{workProblems}

\begin{workToDo}
	\begin{itemize}
		\item Datenstruktur �berlegen.
		\item L�sung f�r Problem (3) finden.
		\item �berlegen wir wir die oben beschriebene Matrix implementieren (einfach als Matrix ergibt Probleme).
		\item Bewertungsoperator umsetzen.
	\end{itemize}
\end{workToDo}
