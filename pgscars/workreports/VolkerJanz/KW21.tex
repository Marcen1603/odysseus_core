\week
\begin{workDone}
	\begin{itemize}
		\item Gating-Algorithmus verstanden.
		\item In vorhandene ansatzweise Umsetzung der Objektverfolgung in Odysseus eingearbeitet.
		\item Gating-Algorithmus komplett implementiert (Mahalanobis-Distanz).
	\end{itemize}
\end{workDone}

\begin{workProblems}
	\begin{itemize}
		\item Die vorhandene ansatzweise Umsetzung der Objektverfolung in Odysseus war zun�chst unklar.
		\item Die Trennung von Operator und Algorithmus wurde zunehmend unklarer und sollte definiert werden.
		\begin{itemize}
			\item Beispiel Gating: der Algorithmus gibt momentan nur true oder false zur�ck -> er vergleicht 1 Paar von Objekten und sagt ob diese verkn�pft werden oder nicht. Das bedeutet, dass der Operator einen Schleifenaufruf haben muss, der den Algorithmus f�r jedes Objektpaar aufruft. Diese Anforderung war zun�chst unklar -> die Schleife k�nnte man auch in den Algorithmus versetzen. Wir haben uns zun�chst darauf geeinigt das der Operator f�r den Aufruf f�r jedes Objektpaar zust�ndig ist.
			\end{itemize}
	\end{itemize}
\end{workProblems}

\begin{workToDo}
	\begin{itemize}
		\item Assoziations-Algorithmus implementieren: wahrscheinlich Nearest-Neighbor.
		\item Filter-Algorithmus implementieren (\textit{Anmerkung: es gibt bereits ausgereifte offene libs die den Kalman-Filter implementieren. Wir sollten uns: http://ai.stanford.edu/\~{}paskin/slam/ genauer ansehen})
		\item J-Unit-Tests f�r die Algorithmen erstellen.
	\end{itemize}
\end{workToDo}
