%%Hier wird das Dokument definiert

\documentclass{scrartcl}  %Format und Stil

\usepackage{../style/wrstyle}
	
% Einstellungen f\"{u}r die "offizielle" Titelseite
\title{Projektgruppe StreamCars} % Titel der Dissertation
\subtitle{w�chentliche Arbeitsberichte}
\author{Hauke Neemann} % Name des Autors
\date{\today}
\surveyor{Andr� Bolles}
\term{Sommersemester 2010 - Wintersemester 2010/11}

%%Hier wird das Dokument definiert
\begin{document}

	\maketitle
	
	\newpage
	
	\tableofcontents
	
% ------------------------------ Wochenbericht KW 20 ------------------------------
\week

\begin{workDone}
	\begin{itemize}
		\item Projektmanagement
		\begin{compactitem}
			\item Mittwochs mit Timo getroffen
			\item Neueste Fortschritte in MSP eingetragen
			\item Aufgaben auf die Teammitglieder verteilt
			\item TOP-Listen f�r Treffen angefertigt
		\end{compactitem}
		\item StreamCars
		\begin{compactitem}
			\item angefangen UDP-Verbindung von JDVE zu ODYSSEUS zu implementieren
		\end{compactitem}
	\end{itemize}
\end{workDone}

\begin{workProblems}
	\begin{itemize}
		\item Strukturen der Daten nach dem Verschicken sind nicht so leicht nachzuvollziehen.
	\end{itemize}
\end{workProblems}

\begin{workToDo}
\begin{itemize}
		\item Projektmanagement
		\begin{compactitem}
			\item Mittwochs mit Timo getroffen
			\item Neueste Fortschritte in MSP eingetragen
			\item Aufgaben auf die Teammitglieder verteilt
			\item TOP-Listen f�r Treffen angefertigt
		\end{compactitem}
		\item StreamsCars
		\begin{compactitem}
			\item mit Sven das neue JDVE zum Laufen bringen.
			\item die UDP-Verbindung weiter implementieren.
		\end{compactitem}
	\end{itemize}
\end{workToDo}

% ------------------------------ Wochenbericht KW 21 ------------------------------
\week
\begin{workDone}
	\begin{itemize}
		\item Projektmanagement
		\begin{compactitem}
			\item Mittwochs mit Timo getroffen
			\item Neueste Fortschritte in MSP eingetragen
			\item Aufgaben auf die Teammitglieder verteilt
			\item TOP-Listen f�r Treffen angefertigt
		\end{compactitem}
		\item StreamsCars
		\begin{compactitem}
			\item versucht mit Sven das neue JDVE zum Laufen zu bringen.
			\item UDP: 50 Autos �bertragen
		\end{compactitem}
	\end{itemize}
\end{workDone}

\begin{workProblems}
	\begin{itemize}
		\item das alte JDVE l�uft nicht stabil (Dominion\_ED st�rzt h�ufiger ab, bzw. startet nicht richtig)
		\item das neue JDVE l�uft noch gar nicht (Dominion\_ED st�rzt h�ufiger ab, bzw. startet nicht richtig, NextGenViewer startet nicht, etc.)
	\end{itemize}
\end{workProblems}

\begin{workToDo}
	\begin{itemize}
		\item Projektmanagement
		\begin{compactitem}
			\item Mittwoch mit Timo treffen
			\item Aufgabenverteilung und Zeiten planen
		\end{compactitem}
		\item UDP-Verbindung
		\begin{compactitem}
			\item Dienstag mit Sven treffen um die UDP-Verbindung auszuarbeiten
		\end{compactitem}
		\item Seminare
		\begin{compactitem}
			\item Seminarvortrag am Freitag in Braunschweig
		\end{compactitem}
	\end{itemize}
\end{workToDo}

% ------------------------------ Wochenbericht KW 22 ------------------------------
\week
\begin{workDone}
	\begin{itemize}
		\item Projektmanagement
		\begin{compactitem}
			\item Mittwochs mit Timo getroffen
			\item Neueste Fortschritte in MSP eingetragen
			\item Aufgaben auf die Teammitglieder verteilt
			\item TOP-Listen f�r Treffen angefertigt
		\end{compactitem}
		\item StreamsCars
		\begin{compactitem}
			\item versucht mit Sven das neue JDVE zum Laufen zu bringen. Nach Konsultation von Ulf Noyer das neue JDVE mit Hilfe der alten Version zum Laufen gebracht.
			\item UDP: die bisherige Testklasse zu einer vern�nftigen Hilfsklasse f�r die Quelle refactort.
			\item eine eigene Anwendung f�r JDVE mittels Baal erstellt.
			\item bei Timo und Thomas in der Findungs-/Diskussionsphase beim Aufstellen der Grammatik f�r die logischen Operatoren mitgearbeitet
		\end{compactitem}
	\end{itemize}
\end{workDone}

\begin{workProblems}
	\begin{itemize}
		\item das alte JDVE l�uft nicht stabil (Dominion\_ED st�rzt h�ufiger ab, bzw. startet nicht richtig)
		\item das neue JDVE lief gar nicht (Dominion\_ED st�rzt h�ufiger ab, bzw. startet nicht richtig, NextGenViewer startet nicht, etc.)
	\end{itemize}
\end{workProblems}

\begin{workToDo}
	\begin{itemize}
		\item Projektmanagement
		\begin{compactitem}
			\item Mittwoch mit Timo treffen
			\item Aufgabenverteilung und Zeiten planen
		\end{compactitem}
	\end{itemize}
\end{workToDo}

% ------------------------------ Wochenbericht KW 23 ------------------------------
\week
\begin{workDone}
	\begin{itemize}
		\item StreamsCars
		\begin{compactitem}
			\item erste Anforderungen gesammelt und strukturiert in die Vorlage von Thomas eingepflegt (Product Backlog)
			\item in Odysseus, speziell Quellenoperatoren, eingearbeitet (wozu die unterschiedlichen Funktionen da sind und wie man diese implementiert)
			\item angefangen aus unserer UDP-Verbindung einen Quellenoperator zu implementieren.
		\end{compactitem}
	\end{itemize}
\end{workDone}

\begin{workProblems}
	\begin{itemize}
		\item Wir haben festgestellt, dass es bereits einen sehr �hnlichen Quellenoperator f�r atomare Daten gibt. 
		\item Wir benutzen f�r den vertikalen Prototypen wahrscheinlich trotzdem weiter unsere Version und sp�ter basteln wir das um, so dass der Operator insgesamt generischer wird.
	\end{itemize}
\end{workProblems}

\begin{workToDo}
	\begin{itemize}
		\item Projektmanagement
		\begin{compactitem}
			\item Mittwoch mit Timo treffen
			\item Aufgabenverteilung und Zeiten planen
		\end{compactitem}
		\item StreamCars
		\begin{compactitem}
			\item den Quellenoperator weiterimplementieren, falls notwendig
			\item auf der JDVE-Seite Ungenauigkeiten f�r die Messungen einbauen, damit die Prediktion richtig arbeiten kann
			\item auf der JDVE-Seite eine initiale Fahrerassistenzfunktion einbauen 
		\end{compactitem}
	\end{itemize}
\end{workToDo}

% ------------------------------ Wochenbericht KW 24 ------------------------------
\week
\begin{workDone}
	\begin{itemize}
		\item Projektmanagement
		\begin{compactitem}
			\item mit Timo das Product Backlog ausgearbeitet
		\end{compactitem}
		\item StreamsCars
		\begin{compactitem}
			\item Probleme mit dem neuen JDVE behoben. Dabei musste der Solution ein UDP-Pr�prozessor hinzugef�gt werden.
			\item Einen Zeitstempel f�r die UDP-�bertragung hinzugef�gt
		\end{compactitem}
	\end{itemize}
\end{workDone}

\begin{workProblems}
	\begin{itemize}
		\item JDVE ben�tigt verschiedene Pr�prozessoren (). Das wussten wir bei der Umstellung auf die neue Version nicht (bzw. haben nicht daran gedacht), so dass uns das viel Zeit gekostet hat.
	\end{itemize}
\end{workProblems}

\begin{workToDo}
	\begin{itemize}
		\item Projektmanagement
		\begin{compactitem}
			\item das Release-Backlog ausarbeiten
			\item Aufgabenverteilung und Zeiten planen
		\end{compactitem}
		\item StreamCars
		\begin{compactitem}
			\item auf der JDVE-Seite Ungenauigkeiten f�r die Messungen einbauen, damit die Prediktion richtig arbeiten kann.
			\item einen Dummy-Operator bauen, damit die Leute auf Odysseusseite testen k�nnen ohne JDVE zu ben�tigen.
		\end{compactitem}
	\end{itemize}
\end{workToDo}
\end{document}
