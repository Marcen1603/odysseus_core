%%Hier wird das Dokument definiert

\documentclass{scrartcl}  %Format und Stil

\usepackage{../style/wrstyle}
	
% Einstellungen f\"{u}r die "offizielle" Titelseite
\title{Projektgruppe StreamCars} % Titel der Dissertation
\subtitle{w�chentliche Arbeitsberichte}
\author{Wolf Bauer} % Name des Autors
\date{\today}
\surveyor{Andr� Bolles}
\term{Sommersemester 2010 - Wintersemester 2010/11}

%%Hier wird das Dokument definiert
\begin{document}

	\maketitle
	
	\newpage
	
	\tableofcontents
	
% Wochenbericht KW 20
\week

\begin{workDone}
	\begin{itemize}
	  \item Pr�sentation f�r Seminar fertig gemacht
		\item Diplomarbeit �ber Broker durchgearbeitet.
		\item Einarbeitung in Objektmodell und Operatoren in Odysseus
		\item Template f�r Arbeitsberichte erstellt.
	\end{itemize}
\end{workDone}

\begin{workProblems}
	\begin{itemize}
		\item Objektmodell kann nicht einfach definiert werden. 
		\begin{compactitem}
			\item Es leitet sich aus den PAF-Funktionen und den zu implementierenden AAS ab. 
			\item Die Modelle werden also durch Input- und Outputschemas der Operatoren definiert.
			\item Die Schemas werden durch "`Create Stream Attribute..."' erzeugt. 
		\end{compactitem}
	\end{itemize}
\end{workProblems}

\begin{workToDo}
	\begin{itemize}
		\item Ausarbeitung f�r Seminar fertig machen
		\item mit Timo Grammatik, logische und physische Opertoren untersuchen
	\end{itemize}
\end{workToDo}

\week

\begin{workDone}
	\begin{itemize}
	  \item Seminar Korrekturen eingearbeitet.
		\item Beginn der Implementierung des ersten Quelloperators. 
		\item UPD Kommunikation testen und einbinden.
	\end{itemize}
\end{workDone}

\begin{workProblems}
	\begin{itemize}
		\item Die Experten f�r die UDP Verbindung waren nicht ansprechbar.
		\item JDVE konnte nicht richtig gestartet werden.
	\end{itemize}
\end{workProblems}

\begin{workToDo}
	\begin{itemize}
		\item Ersten Quelloperator weiter implementieren.
		\item Exkursion Braunschweig. 
	\end{itemize}
\end{workToDo}


\week

\begin{workDone}
	\begin{itemize}
	  \item Exkursion Braunschweig.
		\item Einarbeitung Grammatik, Umwandlung Syntaxbaum in logische Operatoren und Tranformation zu physische.
	\end{itemize}
\end{workDone}

\begin{workProblems}
	\begin{itemize}
		\item Ersten Quelloperator nicht implementieren, da erst Grammatik und logische Operatoren fertig sein sollten.
	\end{itemize}
\end{workProblems}

\begin{workToDo}
	\begin{itemize}
		\item Umwandlung Grammatik in logische Operatoren.
	\end{itemize}
\end{workToDo}

\week

\begin{workDone}
	\begin{itemize}
		\item Visitoren f�r CreateStream ausgearbeitet (mit Timo, Thomas).
		\item Visitoren sind fertig mit anh�ngen neuer Quelle im DataDictonary.
	\end{itemize}
\end{workDone}

\begin{workProblems}
	\begin{itemize}
		\item keine.
	\end{itemize}
\end{workProblems}

\begin{workToDo}
	\begin{itemize}
		\item Umwandlung logische Operatoren in physische Operatoren (mit Timo).
		\item Testlauf mit Konsolenausgabe von Daten aus JDVE.
	\end{itemize}
\end{workToDo}

\week

\begin{workDone}
	\begin{itemize}
		\item Workreport Template wurde angepasst.
		\begin{compactitem}
			\item Es werden nun die Daten der Kalenderwoche angezeigt.
		\end{compactitem}
		\item Umwandlung logische Operatoren in physische Operatoren (mit Timo, Thomas).
		\begin{compactitem}
			\item Es kann nun mittels CreateStream eine JDVE-Quelle erstellt werden.
			\item Erzeugen von Tupel anhand des Output Schemas.
		\end{compactitem}
		\item Protokoll schreiben.
	\end{itemize}
\end{workDone}

\begin{workProblems}
	\begin{itemize}
		\item �bertragungsformat von JDBC zu Odysseus ist nicht ausreichend und muss angepasst werden.
		\begin{compactitem}
			\item Werden Listen �bertragen (bspw. Autos) m�ssen diese im Bytestrom markiert werden. Am besten mit Anzahl der Elemente in der Liste.
		\end{compactitem}
	\end{itemize}
\end{workProblems}

\begin{workToDo}
	\begin{itemize}
		\item Anpassen der JDVE-Quelle an angepa�ten Bytestrom (mit Timo und Thomas).
		\item Erstellen der Grammatiken f�r die PAF Opertaoren (mit Timo und Thomas). Erst m�glich, wenn Anforderungen der PAF Gruppe fertig sind.
	\end{itemize}
\end{workToDo}

\end{document}