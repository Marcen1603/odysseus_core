%%Hier wird das Dokument definiert

\documentclass{scrartcl}  %Format und Stil

\usepackage{../style/wrstyle}
	
% Einstellungen f\"{u}r die "offizielle" Titelseite
\title{Projektgruppe StreamCars} % Titel der Dissertation
\subtitle{w�chentliche Arbeitsberichte}
\author{Benjamin Gr�nebast} % Name des Autors
\date{\today}
\surveyor{Andr� Bolles}
\term{Sommersemester 2010 - Wintersemester 2010/11}

%%Hier wird das Dokument definiert
\begin{document}

	\maketitle
	
	\newpage
	
	\tableofcontents
	
% Wochenbericht KW 20
\week

\begin{workDone}
	\begin{itemize}
		\item Pr�sentation f�r Seminar fertig gemacht.
		\item Ausarbeitung f�r Seminar fertig gemacht.
		\item Einarbeitung in Objektmodell und Operatoren in Odysseus
		\item Einarbeitung in Odysseus
			\begin{compactitem}
				\item Thomas Seminararbeit durchgearbeitet
				\item Die Odysseus-Klassen \texttt{RealationalTupel} und andere zur Schemadefinition ben�tigte Klassen "`angeschaut"' (verstanden)
			\end{compactitem}
	\end{itemize}
\end{workDone}

\begin{workProblems}
	\begin{itemize}
		\item Objektmodell kann nicht einfach definiert werden.
			\begin{compactitem}
				\item Es leitet sich aus den PAF-Funktionen und den zu implementierenden AAS ab. 
				\item Die Modelle werden also durch Input- und Outputschemas der Operatoren definiert.
				\item Die Schemas werden durch "`Create Stream Attribute..."' erzeugt. 
			\end{compactitem}
	\end{itemize}
\end{workProblems}

\begin{workToDo}
	\begin{itemize}
		\item Im allgemeinen, neue Aufgaben suchen.
		\item Schemadefinitionen der Daten im Datenfluss bzgl. der einzelnen Operatoren herrausfinden und besprechen
		\item Erste Quelle in Odysseus erstellen, um UDP-Verbindung und Daten in Odysseus zu testen
	\end{itemize}
\end{workToDo}

% Wochenbericht KW 21
\week
\begin{workDone}
	\begin{itemize}
		\item Erster Klassenentwurf f�r physische Quelle in Odysseus erstellt.
		\item Plugin-Projekt \texttt{de.uniol.inf.is.odysseus.scars.base} erstellt
		\item Analogien und Hilfen in dem Odysseus-Projekt \texttt{de.uniol.inf.is.odysseus.objecttracking} gefunden. (Programm Code nachvollzogen)
	\end{itemize}
\end{workDone}

\begin{workProblems}
	\begin{itemize}
		\item Konnten die Quelle leider nicht mit den UDP Daten aus jDVE testen, da Hauke und Sven nicht da waren, und wir das Ding einfach nicht zum laufen bekommen haben, obwohl sie sagten es w�hre fertig.
	\end{itemize}
\end{workProblems}

\begin{workToDo}
	\begin{itemize}
		\item ...
	\end{itemize}
\end{workToDo}

\week
% Wochenbericht KW 22
\begin{workDone}
	\begin{itemize}
		\item Auslesen der richtigen Attribute aus den relationalen Tupeln (Klasse: \texttt{RelationalTuple} und \texttt{MVRelationalTuple}) versucht nachzuvollziehen, um die Daten f�r die Algorithmen bereitzustellen.
		\item Freitag: Exkurs Braunschweig DLR
		\item Mit Volker die Trennung von Odysseus-Operator und Algorithmus diskutiert und analysiert.
	\end{itemize}
\end{workDone}

\begin{workProblems}
	\begin{itemize}
		\item Leider immer noch mangelnde Klarheit �ber das Auslesen der richtigen Attribute in den Operatoren. Die Klasse \texttt{SDFExpression} beschreibt die Funktion die in PQL angegeben wird und enth�lt die Namen der ben�tigten Attribute. Wie kommt sie in den Operator? 
	\end{itemize}
\end{workProblems}

\begin{workToDo}
	\begin{itemize}
		\item Bericht zur Initialisierung des Kontextmodellls schreiben.
	\end{itemize}
\end{workToDo}

\week
% Wochenbericht KW 23
\begin{workDone}
	\begin{itemize}
		\item Am Montag mit Volker Umsetzung von Operatoren eingearbeitet.
		\item Den Bericht f�r die Initialisierung des Kontextmodells fertig geschrieben.
	\end{itemize}
\end{workDone}

\begin{workProblems}
	\begin{itemize}
		\item Volker war Donnerstag leider krank, daher konnten wir nicht zusammen die Operatoren implementieren.
	\end{itemize}
\end{workProblems}

\begin{workToDo}
	\begin{itemize}
		\item Operatoren implementieren, jedenfalls einen, damit alle einmal gesehen haben wie das geht.
	\end{itemize}
\end{workToDo}

% Wochenbericht KW 23
\week
\begin{workDone}
	\begin{itemize}
		\item Grapfische Zeichnung zur Architektur erstellt, um einen �berblick �ber die zu erstellenden Operatoren und deren Ein-/Ausgabedaten zu erhalten. (Jetzt aber schon wieder veraltet)
		\item Wichtige Aspekte bei der Implementierung des Filter-Operators gekl�rt. (z.B. Pr�diktionsfunktionen werden dort f�r die Matrixberechnungen auch ben�tigt (mit Daniel))
		\item Die "`hoffentlich"' letzten L�cken bei der allgemeinen Implementierung von Operatoren f�r objektrelationale Daten geschlossen. Genauer: wie komme ich an das Tupel herran, das die liste der detektierten Objekte enth�lt? L�sung: Name der Liste wird und anderer wichtiger Attribute wie der Zeitstempel werden in der Anfragesprache definiert, und dem logischen Operator als Strings �bergeben, anschlie�end erfolgt mit hilfe des entsprechenden Eingabeschemas die Aufl�sung in indices der entsprechenden Attribute (nun ist jedoch die Aufl�sung das problem!).
		\item Speziellen Container-Klasse erstellt, die als Metadatum im Shema die PredictionFunctions enth�lt (kann vll sp�ter f�r die Filterung mit Hilfmethoden ausgestattet werden, die bei der Matrixberechnung n�tzlich sind).
		\item Mit volker und den anderen �ber die Parametrisierung und aufteilung des assoziations-operators diskutiert. (Volkers Idee: Assoziationsoperator besteht aus einer Reihe von Operatoren oder Funktionen die die Verkn�pfungen zwischen neu und alt mithilfe einer Korrelations-matrix bewerten, und am ende der Reihe werden jeweils die besten bewertungen gew�hlt. Somit, gute Parametrisierung mit Hilfe verschiedener Bewertungs-operatoren (nicht nur gating).
		\item Interface f�r die Bewertungsfunktion (als Parameter f�r die Bewertungsoperatoren) mit den anderen erstellt.
		\item Meinen Senf zum Product baglock dazugegeben.
	\end{itemize}
\end{workDone}

\begin{workProblems}
	\begin{itemize}
		\item Am Montag und Donnerstag war leider die L�cke f�r die allgemeine Implementierung von Operatoren f�r objektrelationale Daten noch nicht geschlossen. 
		\item die Speziellen Container-Klasse die als Metadatum im Shema die PredictionFunctions enth�lt muss noch mit den anderen besprochen werden.
		\item Aufl�sung des Attributnames, bei einem objektrelationalen Schema. (vll. neuen datentyp: "`or"' oder "`complex"'???) woher wei� ich, wo ich im schema nach dem attributnamen suchen muss? K�nnen wir das festlegen? normalerweise sollte das etwas generischer sein. Und was ist, wenn zwei attribute den gleichen namen haben? (kann warscheinlich nicht passieren) Gesetzt den fall man hat den namen der liste, in der lediglich die messobjekte enthalten sind, kann man dann das schema, welches eigentlich ein baum ist, komplett durchsuchen? Dann reicht nicht einfach nur ein index an welcher stelle das attribute(die liste) steht, sondern es muss ein index pfad angegeben werden. ??? 
	\end{itemize}
\end{workProblems}

\begin{workToDo}
	\begin{itemize}
		\item die Speziellen Container-Klasse die als Metadatum im Shema die PredictionFunctions enth�lt muss noch mit den anderen besprochen werden.
		\item die prediction-Op und predicionAssignOp m�ssen endlich fertig werden, und mit Timo und den anderen zur Erstellung der Grammatik besprochen werden.
		\item den anderen die neuen erkenntnisse �ber die implementierung der operatoren (bzgl. der l�cken, siehe workDone) erkl�ren, so das alle anfangen k�nnen die operatoren zu implementieren.
	\end{itemize}
\end{workToDo}

\end{document}
