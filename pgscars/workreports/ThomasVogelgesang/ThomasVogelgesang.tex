%%Hier wird das Dokument definiert

\documentclass{scrartcl}  %Format und Stil

\usepackage{../style/wrstyle}
	
% Einstellungen f\"{u}r die "offizielle" Titelseite
\title{Projektgruppe StreamCars} % Titel der Dissertation
\subtitle{w�chentliche Arbeitsberichte}
\author{Thomas Vogelgesang} % Name des Autors
\date{\today}
\surveyor{Andr� Bolles}
\term{Sommersemester 2010 - Wintersemester 2010/11}

%%Hier wird das Dokument definiert
\begin{document}

	\maketitle
	
	\newpage
	
	\tableofcontents
	
% Wochenbericht KW 20
\week

\begin{workDone}
	\begin{itemize}
		\item Dokumentvorlagen f�r die Anforderungen, Entwurf und Projektabschlussbericht erstellt
		\item Anderen Gruppenmitgliedern Architektur und Funktiosnweise von Odysseus erkl�rt
		\item Pr�sentationsfolien f�r den Vortrag
	\end{itemize}
\end{workDone}

\begin{workProblems}
	\begin{itemize}
		\item Genaue Gliederung f�r die Dokumente ist noch unbekannt. Diese sollte in einem der n�chsten Treffen besprochen werden
	\end{itemize}
\end{workProblems}

\begin{workToDo}
	\begin{itemize}
		\item Erste Vorschl�ge f�r eine Dokumentstrukur entwerfen
		\item Erste Testimplementierungen f�r logische/pjysische Operatoren, PQL, Transformation etc., um herauszufinden wie dieses genau funktioniert
	\end{itemize}
\end{workToDo}

% Wochenbericht KW 21
\week
\begin{workDone}
	\begin{itemize}
		\item Testoperator vollst�ndig implementiert, in Odysseus integriert und getestet, um mit der Integration von Operatoren vertraut zu werden und m�gliche Probleme zu erkennen
		\item Anderen Gruppenmitgliedern Architektur und Funktionsweise von Odysseus erkl�rt
	\end{itemize}
\end{workDone}

\begin{workProblems}
	\begin{itemize}
		\item Es ist noch nicht klar, wie Quellen in PQL zu erstellen sind
	\end{itemize}
\end{workProblems}

\begin{workToDo}
	\begin{itemize}
		\item Vorschlag f�r eine m�gliche Gliederung der Berichte erarbeiten
		\item Herausfinden, wie Quellen in PQL erstellt werden k�nnen
		\item Herausfinden wie Parameter (z.B. Pr�dikate) in PQL verwendet werden
	\end{itemize}
\end{workToDo}

% Wochenbericht KW 22
\week
\begin{workDone}
	\begin{itemize}
		\item Seminarpr�sentation vorbereitet
		\item Zusammen mit Timo CQL-Grammatik f�r \textit{Create Stream}-Befehl erstellt (besitzt noch keine Logik)
	\end{itemize}
\end{workDone}

\begin{workProblems}
	\begin{itemize}
		\item 
	\end{itemize}
\end{workProblems}

\begin{workToDo}
	\begin{itemize}
		\item Vorschlag f�r eine m�gliche Gliederung der Quelle erarbeiten (zur Zeit noch geringe Priorit�t)
		\item PQL-Grammatik um Befehle f�r Operatoren erweitern
		\item In PQL den Visitor f�r \textit{Create Stream}-Befehl implementieren (falls notwenidg entsprechende logische Operatoren usw. erstellen)
	\end{itemize}
\end{workToDo}

% Wochenbericht KW 23
\week
\begin{workDone}
	\begin{itemize}
		\item Vistor f�r PQL-Grammatik f�r CREATE SENSOR implementiert (mit Timo und Wolf)
		\item Erste Gliederung f�r Anforderungsdokument (Zwischenbericht)
		\item Erste Gliederung f�r Entwurfsdokument (Zwischenbericht)
	\end{itemize}
\end{workDone}

\begin{workProblems}
	\begin{itemize}
		\item Die Gliederungen der Dokumente sind nur vorl�ufig und m�ssen daher in einer Gruppensitzung diskutiert und abgestimmt werden
	\end{itemize}
\end{workProblems}

\begin{workToDo}
	\begin{itemize}
		\item Dokumente f�r Backlogs erstellen (R�cksprache mit Hauke erforderlich)
		\item Implementierung des physischen UDP-Access-Operators (falls Hauke und Sven das noch nicht tun, ansonsten ggf. Zusammenarbeit)
		\item Erstellung der Transformationsregeln f�r den UDP-Access-Operator
	\end{itemize}
\end{workToDo}

% Wochenbericht KW 24
\week
\begin{workDone}
	\begin{itemize}
		\item Dokumentvorlagen f�r Backlogs erstellt
		\item Product-Backlog um weitere Anforderungen erweitert
		\item Zusammen mit Timo und Wolf damit angefangen, die Auswertung der UDP Daten entsprechend des im CREATE SENSOR Befehl angegebenen Schemas durchzuf�hren. So sollen die Tupel f�r den Datenstrom entsprechend des Schemas konstruiert und mit Werten gef�llt werden
	\end{itemize}
\end{workDone}

\begin{workProblems}
	\begin{itemize}
		\item Bei Arrays und Records (die z.t. offenbar auch als arrays versendet werden) muss die Anzahl der vesendeten Objekte zuvor �bertragen werden, um den Bytestrom entsprechend auswerten zu k�nnen
		\begin{itemize}
			\item Erste angedachte L�sung: Eintr�ge von Arrays und Records werden immer als elementare Daten (Typ int, double, long, ...) �bertragen. Dabei muss zuvor stets ein int oder double �bertragen werden, der die Anzahl der folgenden Elemente angibt. An dieser Stelle ist eine aber wohl noch eine weitere Absprache mit Sven und / oder Hauke erforderlich
		\end{itemize}
	\end{itemize}
\end{workProblems}

\begin{workToDo}
	\begin{itemize}
		\item Die Verallgemeinerung der Tupelerzeugung weiter implementieren (zusammen mit Timo und Wolf, ggf. Absprach mit Hauke und/oder Sven)
	\end{itemize}
\end{workToDo}

\end{document}
