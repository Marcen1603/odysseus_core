\section{Funktionale Anforderungen}\label{sec:anforderung:funktionale} \index{Anforderung ! funktional}
% Hier wird die Liste mit allen funktionalen Anforderungen an das System aufgef�hrt
% die Anforderungen sollten ggf. inhaltlich und strukturell (durch subsections) noch weiter unterteilt werden
% Eine inhaltlich zusammenh�ngende Menge vpon Anforderungen sollte in einer description Umgebung zusammengefasst werden. Zur Gliederung der Anforderung und Unteranforderungen gibt es die Befehle
%	\req{prefix}
%	\subreq{prefix}
%	\subsubreq{prefix}
%	\subsubsubreq{prefix}
% Die Befehle bekommen jeweils einen Parameter prefix �bergeben, der das Pr�fix der jeweiligen Anforderungsgruppe ist. Entsprechend des gew�hlten Befehls, erfolgt eine automatische Nummerierung der Anforderungen

\begin{description}
  \item[\req{Odysseus}] Das ist eine Beispiel-Anforderung an Odysseus
  \item[\subreq{Odysseus}] Das ist eine Unteranforderung an Odysseus
  \item[\subreq{Odysseus}] Noch eine Unteranforderung an Odysseus
  \item[\subsubreq{Odysseus}] Das ist eine Unterunteranforderung an Odysseus
  \item[\subsubsubreq{Odysseus}] Das ist eine Beispiel-Anforderung auf der tiefsten Verschachtelungsbene an Odysseus
\end{description}

\begin{description}
  \item[\req{JDVE}] Eine Anforderung an JDVE
\end{description}


% Anforderungen f�r die erste Iteration

% A1				Dominion und Odysseus m�ssen miteinander kommunizieren k�nnen
% 	A1.1			JDVE muss Sensordaten vom Auto bekommen (nicht unsere Anforderung?)

% 	A1.2			Es m�ssen Daten von JDVE (Dominion) zu Odysseus �bertragen werden k�nnen
%			A1.2.1			JDVE muss die Daten per UDP senden
%			A1.2.2			Odysseus muss die Daten per UDP empfangen

%		A1.3			Odysseus muss die Daten verarbeiten k�nnen

%		A1.4			Es m�ssen Daten von Odysseus zu JDVE (Dominion) �bertragen werden k�nnen
%			A1.4.1			Odysseus muss die Daten per UDP senden
%			A1.4.2			JDVE muss die Daten per UDP empfangen

