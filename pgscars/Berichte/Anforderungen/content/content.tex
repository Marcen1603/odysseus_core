% Namensschema der Labels:
%
% Kapitel, Unterkapitel usw.: sec:kapitel:abschnitt:unterabschnitt
% Bilder: fig:bildname
% Code-Listings: lst:listingname
%
%-------------------------------------------------------------------------------
%
% Index-Markierungen mit \index{Indexpunkt ! Unterpunkt} erstellen
%
\chapter{Einleitung}\label{sec:einleitung}
%Einf�hrung und �berblick �ber die gesamte Anforderungsdefinition

\section{Zielsetzung des Dokuments}\label{sec:einleitung:zielsetzung}
%Motivation f�r das Dokument
%angestrebte Verwendung des Dokuments
%Bestimmung der Zielgruppe (Welche Leser)?

\section{Vision}\label{sec:einleitung:vision}
%Die Vision gibt einen kurzen �berblick �ber das Projekt und die grundlegenden (unumst��lichen) Ziele und Rahmenbedingungen (ca. 1-2 Seiten):
%Beschreibung der Motivation und Ziele des Projekts
%Beschreibung der vorgesehenen Anwendungsbereiche
%Welceh Vorteile versprcht man sich von einem positiven Projektausgang

\section{Definitionen, Akronyme und Abk�rzungen}\label{sec:einleitung:definitionen}
% Hier werden alle Begriffe, Akronyme und Abk�rzungen definiert. (Zus�tzliche Definitionen im Glossar)

\section{Referenzen}\label{sec:einleitung:refenzen}
% Hinweis auf alle referenzierten Dokumente. Da wir eine Literaturliste im Anhang haben kann dieser Punkt eventuell wegfallen. Eventuell kann an dieser stelle ein Blick auf verwandte Literatur erfolgen
Hier ist ein Dummy-Zitat, damit das Makefile durchl�uft \cite{dummy}

\section{�berblick}\label{sec:einleitung:ueberblick}
%Hier wird eine �bersicht �ber den Inhalt und die Struktur des Dokuments gegeben

%-------------------------------------------------------------------------------

\chapter{Allgemeine Beschreibung}\label{sec:beschreibung}
% �berblick �ber das System und seinen Einsatzbereich
% Beschreibung allgemeiner Systemfaktoren

\section{Produkt-Einbettung}\label{sec:beschreibung:einbettung}
% Einordnung des Systems in seine Systemumgebung (evtl. Blockdiagramm)
% �bertragung der Anforderungen des Gesamtsystems auf das Teilsystem
% Beschreibungder Schnittstellen des Systems
%	Benutzungsschnittstellen (GUI Design, Funktionstasten, Benutzergruppen)
%	Hardwareschnittstellen (Hardwarekonfiguration, Speicher, unterst�tzte Ger�te)
%	Kommunikationsschnittstellen (LAN Protokolle)
%	Softwareschnittstellen (Komponente, Version, Zweck, Schnittstellenformat)

\section{Produkt-Funktionen}\label{sec:beschreibung:funktionen}
% Zusammenfassung der Funktionen, die durch das Softwaresystem unterst�tzt werden soll.

\section{Anforderungen an die Benutzer}\label{sec:beschreibung:benutzer}
% Beschreibung allgemeiner Anforderungen an die intendierten Anwender (Ausbildung, Erfahrung, technischer Hintergrund)
% Kann eventuell wegfallen, da wir einen wissenschaftlichen Prototyp entwickeln und kein System f�r den operativen Einsatz

\section{Annahmen und Abh�ngigkeiten}\label{sec:beschreibung:abhaengigkeiten}
% Auflistung von Annahmen �ber die Systemumgebung, deren �nderung die Systemanforderungen deutlich �ndern kann (Betriebssystem, speziell zu entwickelnde Hardware)
% Wird auf unser Projekt wahrscheinlich nicht zutreffen

\section{Weitere Anforderungen}\label{sec:beschreibung:weitere}
% Auflistung von Anforderungen, die in einer zuk�nftigen Version realisert werden sollen
% Dieser Abschnitt wird f�r unser Projekt wahrscheinlich nicht erforderlich sein. Evtl. kann man an dieser Stelle einen Ausblick auf m�gliche Entwicklungen geben, die �ber den Umfang der PG hinausgehen

%-------------------------------------------------------------------------------

\chapter{Anforderungen}\label{sec:anforderung} \index{Anforderung}
% detaillierte Beschreibung aller Anforderungen

\section{Funktionale Anforderungen}\label{sec:anforderung:funktionale} \index{Anforderung ! funktional}
% Hier wird die Liste mit allen funktionalen Anforderungen an das System aufgef�hrt
% die Anforderungen sollten ggf. inhaltlich und strukturell (durch subsections) noch weiter unterteilt werden
% Eine inhaltlich zusammenh�ngende Menge vpon Anforderungen sollte in einer description Umgebung zusammengefasst werden. Zur Gliederung der Anforderung und Unteranforderungen gibt es die Befehle
%	\req{prefix}
%	\subreq{prefix}
%	\subsubreq{prefix}
%	\subsubsubreq{prefix}
% Die Befehle bekommen jeweils einen Parameter prefix �bergeben, der das Pr�fix der jeweiligen Anforderungsgruppe ist. Entsprechend des gew�hlten Befehls, erfolgt eine automatische Nummerierung der Anforderungen

\begin{description}
  \item[\req{Odysseus}] Das ist eine Beispiel-Anforderung an Odysseus
  \item[\subreq{Odysseus}] Das ist eine Unteranforderung an Odysseus
  \item[\subreq{Odysseus}] Noch eine Unteranforderung an Odysseus
  \item[\subsubreq{Odysseus}] Das ist eine Unterunteranforderung an Odysseus
  \item[\subsubsubreq{Odysseus}] Das ist eine Beispiel-Anforderung auf der tiefsten Verschachtelungsbene an Odysseus
\end{description}

\begin{description}
  \item[\req{JDVE}] Eine Anforderung an JDVE
\end{description}


% Anforderungen f�r die erste Iteration

% A1				Dominion und Odysseus m�ssen miteinander kommunizieren k�nnen
% 	A1.1			JDVE muss Sensordaten vom Auto bekommen (nicht unsere Anforderung?)

% 	A1.2			Es m�ssen Daten von JDVE (Dominion) zu Odysseus �bertragen werden k�nnen
%			A1.2.1			JDVE muss die Daten per UDP senden
%			A1.2.2			Odysseus muss die Daten per UDP empfangen

%		A1.3			Odysseus muss die Daten verarbeiten k�nnen

%		A1.4			Es m�ssen Daten von Odysseus zu JDVE (Dominion) �bertragen werden k�nnen
%			A1.4.1			Odysseus muss die Daten per UDP senden
%			A1.4.2			JDVE muss die Daten per UDP empfangen



%Hier werden alle nichtfunktionalen Anforderungen aufgef�hrt. Dabei werden die nichtfunktionalen Anforderungen inhaltlich sortiert:
\section{Leistungsanforderungen}\label{sec:anforderung:leistung} \index{Anforderung ! Leistung}
%numerische Anforderungen an das System oder die Interaktion mit dem System

\section{Schnittstellenanforderungen}\label{sec:anforderung:schnittstelle} \index{Anforderung ! Schnittstelle}
% detaillierte Beschreibung aller Ein- und Ausgaben

\section{Entwurfsanforderungen}\label{sec:anforderung:entwurf} \index{Anforderung ! Entwurf}
% Anforderungen an den Softwareentwurf, z.B. durch Standards
% An dieser Stelle w�rde man dann wohl auch Scrum erw�hnen

\section{Quialit�tsanforderungen}\label{sec:anforderung:qualitaet} \index{Anforderung ! Qualit�t}
% Anforderungen an Softwarequalit�t (Zuverl�ssigkeit, Verf�gbarkeit, Sicherheit, Wartbarkeit, Portabilit�t)

\section{Weitere Anforderungen}\label{sec:anforderung:weitere} \index{Anforderung ! Weitere}

\chapter{Zusammenfassung}\label{sec:zusammenfassung}