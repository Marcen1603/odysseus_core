% Namensschema der Labels:
%
% Kapitel, Unterkapitel usw.: sec:kapitel:abschnitt:unterabschnitt
% Bilder: fig:bildname
% Code-Listings: lst:listingname
%
%-------------------------------------------------------------------------------
%
% Index-Markierungen mit \index{Indexpunkt ! Unterpunkt} erstellen
%
\chapter{Einleitung}\label{sec:einleitung}

\section{Zweck des Systems}\label{sec:einleitung:zweck}

\section{Entwurfsziele}\label{sec:einleitung:entwurfsziel}

\section{Definitionen, Akronyme und Abk�rzungen}\label{sec:einleitung:definitionen}

\section{Referenzen}\label{sec:einleitung:referenzen}
Hier ist ein Dummy-Zitat, damit das Makefile durchl�uft \cite{dummy}
\section{�berblick}\label{sec:einleitung:ueberblick}

%-------------------------------------------------------------------------------

\chapter{Aktuelle Softwarearchitektur}\label{sec:aktuell}
% In diesem Kapitel wird auf die bereits vorhandenen Systeme eingegangen. Das sind in unserem Fall vor allem Odysseus und JDVE

\section{Odysseus}\label{sec:aktuell:odysseus}

\section{JDVE}\label{sec:aktuell:jdve}

%-------------------------------------------------------------------------------

\chapter{Vorgeschlagene Softwarearchitektur}\label{sec:architektur}
% Hier wird der eigentliche Entwurf pr�sentiert

\section{�berblick}\label{sec:architektur:ueberblick}
% hier wird eine �bersicht �ber den weiteren Aufbau des Kapitels gegeben

\section{Systemzerlegung}\label{sec:architektur:systemzerlegung}
% Hier erfolgt eine Unterteilung des Systems in Subkomponenten

\section{Abbildung auf Hardware-/Software-Komponenten}\label{sec:architektur:komponentenabbildung}
% Hier wird beschrieben welche Komponenten auf welchen Hardware bzw. Software Plattformen betrieben werden sollen

\section{Globaler Kontrollfluss}\label{sec:architektur:kontrollfluss}
% Hier wird beschrieben, wie die Komponenten unterienander interagieren, wo die Einstiegspunkte ins System sind und wie die Arbeitsvorg�nge sequentialisiert werden

\section{Randbedingungen}\label{sec:architektur:randbedingungen}
% Hier werden die randbedingungen erl�utert, die f�r das System erf�llt sein m�ssen: Initialisierung und Herunterfahren des Systems, Handhabung von Ausnahmef�llen

%-------------------------------------------------------------------------------

\chapter{Subsystemdienste}\label{sec:subsystemdienste}
% Hier werden noch einmal alle Subsysteme aufgef�hrt und deren Funktionalit�t kurz beschrieben. Wozu man das noch einmal braucht wusste schon im SWP niemand ;)

%-------------------------------------------------------------------------------

\chapter{Zusammenfassung}\label{sec:zusammenfassung}