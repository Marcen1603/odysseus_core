\documentclass[pdftex,a4paper]{scrartcl}
\usepackage{ngerman}
\usepackage[latin1]{inputenc}
\usepackage[T1]{fontenc}
\usepackage[inner=2cm,outer=2cm,top=2cm,bottom=2cm,includeheadfoot]{geometry}
\usepackage{rotating}
\usepackage{paralist}
\usepackage{url}
\setlength{\evensidemargin}{\paperwidth}
\addtolength{\evensidemargin}{-\textwidth}
\addtolength{\evensidemargin}{-2.0in}
\addtolength{\evensidemargin}{-\oddsidemargin}
\setlength{\parindent}{0pt}
%\renewcommand{\labelenumi}{\alph{enumi})}
\parskip 12pt


\usepackage{fancyhdr}
\pagestyle{fancy}
\fancyhf{}
\fancyhead[L]{Sitzungsprotokoll}
%Sitzungsdatum
\fancyhead[R]{22.04.2010}
\begin{document}

\renewcommand{\headrulewidth}{0.5pt}

\fancyfoot[C]{\thepage}
\fancyfoot[R]{\today}
\renewcommand{\footrulewidth}{0.5pt}

\begin{center}
\sc
%X-te Sitzung
\LARGE{Protokoll zur 2. Sitzung} \\
%Sitzungsdatum
\Large{\textbf{am 22.04.2010}} \\
%Uhrzeit
\normalsize{\textbf{um 14:15 - 16:00}}
\end{center}
%Moderator bzw Sitzungsleiter
\begin{large}\textbf{Sitzungsleiter:} Andr� Bolles\\
%Protokolant
\textbf{Protokolf"uhrer:} Wolf Bauer\\
%Wer anwesend ist und wer nicht
\textbf{Anwesend:} Wolf Bauer, Andr� Bolles, Benjamin Gr�nebast, Volker Janz, Nico Klein, Tobias Krahn, Timo Michelsen, Sven M�ller, Hauke Neemann, Daniel Twumasi, Thomas Vogelgesang
\newline

\end{large}
\rule{\textwidth}{0.2pt}
\rm


%~~~+++ Tagesordnungsvorstellung +++~~~
% Hier steht zun�chst die vorgeschlagene Tagesordnung f�r die Sitzung
\chapter{Tagesordnung}
 \begin{itemize}
	\item Briefing
	\item Dokumentenverwaltung
	\item Projektmanagement
	\item JDBI
 \end{itemize}

%~~~+++ Briefing +++~~~
% Hier steht ein Briefing, sofern durchgef�hrt
\section{Briefing}
	\begin{itemize}
		\item Jan Sattler ist aus der Projektgruppe ausgetreten.
		\item JDBI ist im SWL eingerichtet (siehe JDBI Abschnitt).
   \end{itemize}

%~~~+++ N�chster Punkt  +++~~~ 
\section{Dokumentenverwaltung}
\begin{itemize}
	\item SVN f�r Dokumente wird eingerichtet.
	\item Quellcode kommt direkt ins Odysseus SVN.
	\item Zugangsdaten f�r das Odysseus SVN:
	\begin{compactitem}
		\item Adresse: http://isdb1.offis.uni-oldenburg.de/repos/odysseus/trunk
		\item Zugangsdaten f�r Hans Meier: 
		\begin{compactitem}
			\item Nutzername: Hmeier evtl. auch hmeier
			\item Passwort: hmeier\#\#
		\end{compactitem}
	\end{compactitem}
	\item Hier \url{http://javathreads.de/2008/07/subversion-unter-eclipse-ganymede-konfigurieren/} gibt es ein Tutorial f�r ein Eclipse SVN PlugIn.
	\item Odysseus besteht aus vielen Projekten. Zum Auschecken kann der Eclipse SVN Browser genutzt werden.
\end{itemize}

\section{Remote-Desktop zum SWL}
\begin{itemize}
	\item Mit Hilfe des Remote Desktop kann auf die SWL Rechner zugegriffen werden.
	\item Zugangsdaten:
	\begin{compactitem}	
		\item IP: 134.106.56.39
		\item Nutzername: pgstreamcars
		\item Passwort: kotelett
	\end{compactitem}
\end{itemize}

\section{Projektmanagement}
\subsection{Einf�hrung MS Project}
\begin{itemize}
	\item Andr� gibt f�r Hauke und Timo eine kurze Einf�hrung in MS Project (Inhalt siehe letztes Protokoll).
	\item Das Programm funktioniert m��ig.
	\item Abstimmung: MS Projekt wird nicht als Projektmanagementtool verwendet.
\end{itemize}

\subsection{OSE}
\begin{itemize}
	\item Andr� erkl�rt kurz eine Technik zum Projektmanagment.
	\item Wird nicht durch eine Software unterst�tzt.
	\item Andr� stellt Dokument dazu bereit.
\end{itemize}

\subsection{Ergebnis}
\begin{itemize}
	\item Wir nutzen vermutlich Trac.
	\item Muss noch genau abgestimmt werden.
\end{itemize}
%~~~+++ N�chster Punkt  +++~~~ 
\section{JDBI}
\begin{itemize}
	\item JDBI liegt in einem Ordner auf dem Desktop auf SWL PCs.
	\item Die Konfiguration wird in TestScenarioExpressway.bat (Sensoren, Views etc.) festgelegt.
	\item Das Fahrerassistenzsystem (FAS) hei�t PADAS.
	\item Das FAS kann ist eine VS C++ Solution.
	\item Das f�r uns wichtige Projekt hei�t \texttt{ISiPADAS\_PADAS}.
	\item Zur Simulation muss in diesem Projekt die \texttt{DISiPADAS.cpp:run} Methode angepasst werden. Dieser Code ist generiert worden und man kann auf alle Fahrzeuge und Sensoren zugreifen, die bei der Codeerzeugung ber�cksichtig wurden. Wir haben einen Frontradarsensor zur Verf�gung. Bei �nderungsw�nschen (neue Sensoren) m�ssen durch das DLR eingeplegt werden.
	\item Eingaben die in der Simulation genutzt werden k�nnen:
	\begin{compactitem}
		\item Umgebungsinformationen: \texttt{\_input.Environment}
		\item Sensordaten: \texttt{\_input.Envirnment.SensorData}
		\item erfa�te Fahrzeuge: \texttt{\_input.Envirnment.SensorData.DetectedVehicles[]}
		\item Eigenes Fahrzeug: \texttt{\_input.Traffic.Vehicles[0]}
	\end{compactitem}
	\item Ausgaben zur Manipulation des Fahrzeugs, die in der Simulation genutzt werden k�nnen:
	\begin{compactitem}
		\item Cockpit beeinflussen: \texttt{\_output.HMI.Cockpit}
		\item Schneller fahren (in Prozent): \texttt{\_output.HMI.Cockpit.Pedals.ThrottleAct = 100};
	\end{compactitem}
	\item Logiken um Aktuatoren zu beeinflu�en (bspw. sanftes Abbremsen) kommt in JDBI (c++) nicht nach Odysseus (Java).
	\item Kommunikation sollte per UDP erfolgen (wie Streaming, verbindungslose Kommunikation).
	\item Im SWL sollte nur VS 2005 genutzt werden.
\end{itemize}

%~~~+++ Arbeitsauftr�ge +++~~~
\section{Arbeitsauftr�ge}
\begin{itemize}
	\item Projektmanagement sammelt alle Kontaktdaten und reicht sie an alle Gruppenmitgleider.
	\item N�chste Woche erste Version der Seminare abgeben.
	\item Wolf bereitet f�r n�chste Sitzung die Trac Integration in Eclipse vor.
\end{itemize}

%~~~+++ N�chste Sitzung +++~~~
\section{N�chste Sitzung}
	\begin{tabular}{|c|c|c|c|c|}
		\hline 
		Datum & Zeit & Ort & Sitzungsleiter & Protokollf"uhrer\\
		\hline 
		Tag, 29.04.2010 & 14:00 - 16:00 & OFFIS U61 & ?? & ??	\\
		\hline
	\end{tabular}

\end{document}