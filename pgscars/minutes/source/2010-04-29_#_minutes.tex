\documentclass[pdftex,a4paper]{scrartcl}
\usepackage{ngerman}
\usepackage[latin1]{inputenc}
\usepackage[T1]{fontenc}
\usepackage[inner=2cm,outer=2cm,top=2cm,bottom=2cm,includeheadfoot]{geometry}
\usepackage{rotating}
\usepackage{paralist}
\usepackage{url}
\setlength{\evensidemargin}{\paperwidth}
\addtolength{\evensidemargin}{-\textwidth}
\addtolength{\evensidemargin}{-2.0in}
\addtolength{\evensidemargin}{-\oddsidemargin}
\setlength{\parindent}{0pt}
%\renewcommand{\labelenumi}{\alph{enumi})}
\parskip 12pt


\usepackage{fancyhdr}
\pagestyle{fancy}
\fancyhf{}
\fancyhead[L]{Sitzungsprotokoll}
%Sitzungsdatum
\fancyhead[R]{29.04.2010}
\begin{document}

\renewcommand{\headrulewidth}{0.5pt}

\fancyfoot[C]{\thepage}
\fancyfoot[R]{\today}
\renewcommand{\footrulewidth}{0.5pt}

\begin{center}
\sc
%X-te Sitzung
\LARGE{Protokoll zur 3. Sitzung} \\
%Sitzungsdatum
\Large{\textbf{am 29.04.2010}} \\
%Uhrzeit
\normalsize{\textbf{um 14:15 - 16:00}}
\end{center}
%Moderator bzw Sitzungsleiter
\begin{large}\textbf{Sitzungsleiter:} Andr� Bolles\\
%Protokolant
\textbf{Protokolf"uhrer:} Benjamin Gr�nebast\\
%Wer anwesend ist und wer nicht
\textbf{Anwesend:} Wolf Bauer, Andr� Bolles, Benjamin Gr�nebast, Volker Janz, Nico Klein, Tobias Krahn, Timo Michelsen, Sven M�ller, Hauke Neemann, Daniel Twumasi, Thomas Vogelgesang
\newline

\end{large}
\rule{\textwidth}{0.2pt}
\rm


%~~~+++ Tagesordnungsvorstellung +++~~~
% Hier steht zun�chst die vorgeschlagene Tagesordnung f�r die Sitzung
\chapter{Tagesordnung}
 \begin{itemize}
	\item Einf�hrung Trac (Projektverwaltung) von Wolf Bauer
	\item Vorstellung von Vorgehensmodell(Projektplanung) Scrum von Hauke Neemann
	\item grundsetzliche Anforderungen an das Projekt von Andr� Boolles
 \end{itemize}

\section{Einf�hrung Trac von Wolf Bauer}
	\begin{itemize}
		\item Trac bietet ein Ticketverwaltungssytem f�r unsere Aufgabenverteilung. Dabei kann jeder von uns Tickets erstellen, anderen Personen zuweisen, sowie Tickets zur Bearbeitung annehmen.
		\item Link zum Trac des Odysseus-Projekts: http://isdb1.offis.uni-oldenburg.de:8000/odysseus/
		\item Logindaten: die gleichen wie Odysseus-SVN-Login
   \end{itemize}
   \subsection{Trac Ticketverwaltung}
   	\begin{itemize}
			\item Bei der Erstellung neuer Tickets soll in dem Formular als version: odysseus\_osgi verwendet werden.
			\item die Componenten m�ssen wir selber ausdenken bzw. aussuchen
			\item jedem Ticket k�nnen Dateien, wie z.B. source-code oder konsolenausgaben hinzugef�gt werden
		\end{itemize}
		\subsection{Integration in Eclipse - Installation}
		Folgende Update-Sites in Eclipse hinzuf�gen (unter: Help->install new Software):
		\begin{itemize}
			\item mylyn update seite: http://download.eclipse.org/tools/mylyn/update/e3.4
			\item mylyn extras seite: http://download.eclipse.org/tools/mylyn/update/extras
		\end{itemize}
		Folgende Plugins von diesen Seiten in eclipse installieren:
				\begin{itemize}
			\item Mylyn Task List (aktuell Version 3.3.3...)
			\item Mylyn Connector: Trac (aktuell Version 3.3.3...)
		\end{itemize}
		\subsection{Integration in Eclipse - Configuration}
		In der Perspective "`Planning"' l�sst sich neben der "`Task List View"' die View "`Task Repositories View"' �ffnen und anzeigen. In der Repositories View k�nnen nun neue Repositories hinzuge�gt werden, wo als erstes "`Trac"' als "`Repository Type"' ausgew�hlt werden muss. Anschlie�end den oben genannten Link unter Server eintragen und die account daten eingeben. Wichtig hier, als erstes m�ssen die Einstellungen mittels "`validate Settings"' validiert werden. Danach sollte alles funktionieren, mit "`Finish"' beenden.
		
		In der "`Task List View"' lassen sich nun neben neuen Tickets auch neue Queries anlegen, mit deren Hilfe nach bestimmten Tickets gesucht werden kann und in der Lista danach gefiltert werden.
		
		    
%~~~+++ N�chster Punkt  +++~~~ 
\section{Vorgehensmodell der Projektplanung Scrum von Hauke Neemann}
Als Vorgehensmodell f�r die Projektplanung soll Scrum genutzt werden. Dieses Modell wurde von Hauke Neemann grob vorgestellt.

\section{Anforderungen von Andr� Bolles}
Andr� hat die grundlegenden Anforderungen grob aufgezeigt, was das Projekt am ende k�nnen muss.
\begin{itemize}
	\item Prim�r: Die Simulationssoftware an Odysseus anbinden
	\item vom Sensor erkannte Objekte einlesen k�nnen
	\item Die Zugeh�rigkeit der Objekte zu einem Scanzyklus muss gegeben sein
	\item nicht jedes Objekt einzelnd im Datenstrom sondern eine Liste mit allen im Zyklus erkannten Objekten auf einmal
	\item Objektvervolgung (vermutlich aufteilen in drei Datenstrom-Operatoren):
		\begin{itemize}
			\item Pr�dektion: Zeitbezug / wo das Objekt im n�chsten Zeitpunkt sein k�nnte.
			\item Assoziation: Verkn�pfen alter Messungen mit den neuen. Sprich: Welches Objekt der alten Messung ist Objekt der neuen Messung? (hier reicht nearest neighbor algorithm)
			\item Filterung: Was bzw. Messabweichungen / Wo ist der wirkliche neue Wert des Objektes? (hier reicht einfacher Kalmann-Filter)
		\end{itemize}
		\item Der Broker-Operator wird aus der Objektverfolgung mit den richtigen Daten gef�ttert, und stellt im gewissen Ma�e unser Umgebungsmodell dar.
	\item alles soll auf objekt-relationaler Ebene entwickelt werden, ohne Ber�cksichtigung auf die Implementierung im richtigem Auto. D.h. auch Performance im Bezug zum richtigem Auto f�r uns erstmal irrelevant.
\end{itemize}
Eine detailiertere Beschreibung der Anforderungen kann in Andr�'s Dissertation gefunden werden. 

\section{Sonstiges Wichtiges}


%~~~+++ Arbeitsauftr�ge +++~~~
\section{Arbeitsauftr�ge}
\begin{itemize}
	\item Odysseus SVN und Trac einrichten, und in Eclipse IDE installieren.
\end{itemize}

%~~~+++ N�chste Sitzung +++~~~
\section{N�chste Sitzung}
	\begin{tabular}{|c|c|c|c|c|}
		\hline 
		Datum & Zeit & Ort & Sitzungsleiter & Protokollf"uhrer\\
		\hline 
		Montag, 03.05.2010 & 14:00 - 16:00 & OFFIS U61 & Andr� Bolles & ??	\\
		\hline
	\end{tabular}
\end{document}