\documentclass[pdftex,a4paper]{scrartcl}
\usepackage{ngerman}
\usepackage[latin1]{inputenc}
\usepackage[T1]{fontenc}
\usepackage[inner=2cm,outer=2cm,top=2cm,bottom=2cm,includeheadfoot]{geometry}
\usepackage{rotating}
\usepackage{paralist}
\usepackage{url}
\setlength{\evensidemargin}{\paperwidth}
\addtolength{\evensidemargin}{-\textwidth}
\addtolength{\evensidemargin}{-2.0in}
\addtolength{\evensidemargin}{-\oddsidemargin}
\setlength{\parindent}{0pt}
%\renewcommand{\labelenumi}{\alph{enumi})}
\parskip 12pt


\usepackage{fancyhdr}
\pagestyle{fancy}
\fancyhf{}
\fancyhead[L]{Sitzungsprotokoll}
%Sitzungsdatum
\fancyhead[R]{17.06.2010}
\begin{document}

\renewcommand{\headrulewidth}{0.5pt}

\fancyfoot[C]{\thepage}
\fancyfoot[R]{\today}
\renewcommand{\footrulewidth}{0.5pt}

\begin{center}
\sc
%X-te Sitzung
\LARGE{Protokoll zur 9. Sitzung} \\
%Sitzungsdatum
\Large{\textbf{am 17.06.2010}} \\
%Uhrzeit
\normalsize{\textbf{um 14:00 - 16:00}}
\end{center}
%Moderator bzw Sitzungsleiter
\begin{large}\textbf{Sitzungsleiter:} Timo Michelsen\\
%Protokolant
\textbf{Protokolf"uhrer:} Wolf Bauer\\
%Wer anwesend ist und wer nicht
\textbf{Anwesend:} Andr� Bolles, Benjamin Gr\"{u}nebast, Volker Janz,  Tobias Krahn, Timo Michelsen, Sven M\"{u}ller, Daniel Twumasi, Thomas Vogelgesang, Nico Klein, Wolf Bauer, Tobias Krahn
\newline

\end{large}
\rule{\textwidth}{0.2pt}
\rm


%~~~+++ Tagesordnungsvorstellung +++~~~
% Hier steht zun�chst die vorgeschlagene Tagesordnung f�r die Sitzung TODO
\chapter{Tagesordnung}
 \begin{itemize}
	\item Briefing
	\item R�ckmeldung zum Produktbacklog
 \end{itemize}

%~~~+++ Briefing +++~~~
% Hier steht ein Briefing, sofern durchgef�hrt
\section{Briefing}
	\begin{itemize}
		\item Keine Berichte.
  	\end{itemize}

\section{R�ckmeldung Produktbacklog}
	\begin{itemize}
		\item  neu: J-1-4-2 Varianzen auch relativ in Prozenten angeben k�nnen.
		\item K-1-4 und K-2-4: Operatoren kapseln Kommunikationsstandard -> Weitere Anforderung die unter Odysseues bei den Operatoren  hinzugef�gt werden muss.
		\item Assoziation und Filterungen k�nnen mehr als nur bin�r sein -> muss in Opertatoren ber�cksichtigt werden.
		\item A-5-3: nicht ganz richtig: es gibt mind. 2 Broker. Muss noch klarer erkl�rt werden.
		\ietm-T-2-4, T-2-6-1, T-2-6-2: ---> T-2-4, T-2-4-1, T-2-4-2 (siehe E-Mail von Andr� zum Backlog).
	\end{itemize}

%~~~+++ Arbeitsauftr�ge +++~~~
\section{Arbeitsauftr�ge}
\begin{itemize}
	\item Thomas, Wolf, Timo arbeiten an Grammatiken und Umwandlung von logischen zu physischen Opeatoren.
	\item Sven und Hauke arbeiten an JDVE Anbindung und Kommunikation 
	\item Benny, Nico, Volker, Daniel und Tobi arbeiten an PAF Opertaoren
\end{itemize}

%~~~+++ N�chste Sitzung +++~~~
\section{N�chste Sitzung}
	\begin{tabular}{|c|c|c|c|c|}
		\hline 
		Datum & Zeit & Ort & Sitzungsleiter & Protokollf"uhrer\\
		\hline 
		Montag, 21.06.2010 & 12:00 - 16:00 & OFFIS U61 & Hauke Neemann & ??\\
		\hline
	\end{tabular}

\end{document}